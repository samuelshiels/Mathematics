\documentclass[10pt,a4paper]{report}
\usepackage[latin1]{inputenc}
\usepackage{amsmath}
\usepackage{amsfonts}
\usepackage{amssymb}
\usepackage{graphicx}

\begin{document}
	\begin{center}
		\underline{\textbf{Partial Differential Equations}}
	\end{center}
	
	\tableofcontents

\chapter{Basics}

A PDE is an equation relating the partial derivatives of some unknown function

https://www.youtube.com/watch?v=atvw5iseoGQ\&list=PLF6061160B55B0203\&index=3

\chapter{Partial Differential Equations}

\section{Notation}
First order differential of u w.r.t t
\begin{equation}
	\frac{du}{dx}
\end{equation}
\begin{equation}
	\dot u
\end{equation}
Second order differential of u w.r.t t
\begin{equation}
	\frac{d^2u}{dx^2}
\end{equation}
\begin{equation}
	\ddot u
\end{equation}
Laplace function
\begin{equation}
	\Delta u (x) = \sum_{i=0}^{n} \frac{d^2u}{dx_i^2} (x) = 0
\end{equation}

\section{Definitions}

\section{Equations}

\subsection{Heat Equation}

The Heat Equation was originally developed to model how heat diffuses across a region

\begin{equation}
	\frac{du}{dt} = \frac{d^2u}{dx^2}
\end{equation}
\begin{equation}
	\dot u = \Delta u
\end{equation}

In this specific instance of the Heat Equation we are working in a single dimension

\subsection*{Example}

Is he following equation a solution to the heat equation

\[ u = \frac{1}{2} x^2 + t \]

Check

\[ \frac{d}{dt} u = \frac{d}{dt}\frac{1}{2}x^2 + \frac{d}{dt} t\]

\[ \frac{d}{dt} u = 1\]

\[ \frac{d^2}{dx^2} u = \frac{d^2}{dx^2}\frac{1}{2}x^2 + \frac{d^2}{dx^2} t\]

\[ \frac{d^2}{dx^2} u = 1\]

\[\therefore \frac{d}{dt} u = \frac{d^2}{dx^2} u \]
\subsection*{Family of Solutions}

The following equation can generate an infinite number of solutions to the heat equation.

\[ u = e ^{ax + bt}\]

We can determine what a and b can represent as follows

\[ \frac{d}{dt} u = e ^{ax + bt} \frac{d}{dt} (ax+bt)\]
\[ = e^{ax + bt}\cdot b\]

\[ \frac{d}{dx} u = e ^{ax + bt}\frac{d}{dx} (ax+bt)\]
\[ = e^{ax + bt}\cdot a\]
\[ \frac{d^2}{dx^2} u=e^{ax + bt}\cdot a^2\]

\[ e^{ax + bt}\cdot b=e^{ax + bt}\cdot a^2\]
\[ \therefore b=a^2\]

\subsection{Wave Equation}

\begin{equation}
	\frac{d^2u}{dt^2} = t^2 \frac{d^2u}{dx^2}
\end{equation}

\subsection*{Derivation}

Model a string being moved up and down in a flat plane. Due to the nature of the string the force acting upon it to return to the y=0 axis is proportional to the "concavity".

Force = F = ma

Since acceleration is the 2nd derivative to position

F = m $\cdot$ $\frac{d^2u}{dt^2}$

Concavity can be modelled as the 2nd derivative of the shape of the string, since we are describing the degree of the curve the string is creating

$\therefore \frac{d^2u}{dx^2}$

Since we now have both sides of the equation but we are missing $c^2$. We can compare the units each side so far.

where c2 is the velocity of the string

\subsection{Laplace Equation}

\subsection{Transport Equation}

\section{Basics}
? Does a solution exist?
\newline
? Is the solution unique?
\newline
? Does the solution depend continuously on the data?
\newline
? How regular is the solution? Is it continuously differentiable? Or even
smooth?
\newline

\end{document}