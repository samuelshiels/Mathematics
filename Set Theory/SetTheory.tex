\documentclass[10pt,a4paper]{report}
\usepackage[latin1]{inputenc}
\usepackage{amsmath}
\usepackage{amsfonts}
\usepackage{amssymb}
\usepackage{graphicx}
\begin{document}
	\begin{center}
		\underline{\textbf{Set, Group, Field and Ring Theory}}
	\end{center}

\tableofcontents

\part{Introduction}
\chapter{History}

\part{Method}
\chapter{Basics}
\section{Simple Sets}
A collection of objects is defined as a Set. The objects have to be distinct and a set is an object itself.
\subsection*{Notation}
We demonstrate a set with the following notation called Roster Notation:
\begin{equation}
S = \{1,2,3\}
\end{equation}
This is useful for small finite sets, but many descriptions will use Set-builder Notation:
\begin{equation}
S = \{n | n \mod2 = 0, 0<n<21\}
\end{equation}
Which describes all the even numbers between 0 and 21.
\subsection*{Membership}
If we have an element of a set we describe it as a member and use the following notation:
\begin{equation}
	x \in B
\end{equation}
We can also describe exclusion from a set:
\begin{equation}
	x \notin B
\end{equation}

\subsection*{Subset}
Further to membership we can compare two sets and determine whether one set is contained within another. Meaning each individual element of set B is at least in the set A. A 'Proper Subset' is described where a subset is not equal to the other set.

\begin{equation}
	\{1,2,3\} \subset \{1,2,3,4\}
\end{equation}
\begin{equation}
	\{1,2\} \subseteq \{1,2\}
\end{equation}
\subsection*{Power Set}

A power set is the set of all subsets of a set.
\begin{equation*}
	S = \{1,2,3\}
\end{equation*}
\begin{equation}
	P(S) = \{\{\varnothing\},\{1\},\{2\},\{3\},\{1,2\},\{1,3\},\{2,3\},\{1,2,3\}\}
\end{equation}

\subsection*{Cardinality}
For a set S, the cardinality is the number of members of S.
\begin{equation*}
	|S| = 3
\end{equation*}
\begin{equation*}
|P(S)| = 8
\end{equation*}
\section{Special Sets}

\begin{math}
\mathbb{P}
\end{math}
the set of all primes
\newline
\begin{math}
\mathbb{N}
\end{math}
the set of all natural numbers
\newline
\begin{math}
\mathbb{Z}
\end{math}
the set of all integers
\newline
\begin{math}
\mathbb{Q}
\end{math}
the set of all rational numbers
\newline
\begin{math}
\mathbb{R}
\end{math}
the set of all real numbers
\newline
\begin{math}
\mathbb{C}
\end{math}
the set of all complex numbers
\newline
\begin{math}
\mathbb{H}
\end{math}
the set of all quaternions

\section{Basic Operations}
\subsection*{Unions}
A union of sets is the set formed by all elements in each set.
\newline Denoted by A $\cup$ B
\subsection*{Intersections}
The intersection of sets is the set formed by the common elements in all sets. 
\newline Denoted by A $\cap$ B
\subsection*{Complements}
A complement is the subtraction of all elements in one set from another.
\newline Denoted by A $\setminus$ B
\subsection*{Cartesian}	
The Cartesian is the set formed by associating every element in one set with every element of the other set.
\newline Denoted by A $\times$ B
\chapter{Group Theory}
\section{Groups}
A Group is a Set equipped with a single operation that combines two elements to form a third that fulfills a series of axioms.
\subsection*{Notation}
We demonstrate a group with the following notation:
\begin{equation}
	G = (S,+)
\end{equation}
Which a Set S with the addition operation to form Group G.

\subsection{Axioms}
To qualify as a group it needs to fulfill the following series of axioms.

\subsection*{Closure}
For any two elements in G, the result of the operation is also in G.
\begin{equation}
	a\cdot b \in G\ \forall\ a,b \in G
\end{equation}

\subsection*{Associativity}
\begin{equation}
	(a \cdot b) \cdot c = a \cdot (b \cdot c)\ \forall\ a,b,c \in G 
\end{equation}

\subsection*{Identity element}
There exists an element in the group such that applying the operation with it against other elements equals that element.
\begin{equation}
	\exists\ i\in G\ s.t. a \cdot i = a 
\end{equation}

\subsection*{Inverse element}
For each element in the group there is another element that when acted together with the operation equals the identity element.
\begin{equation}
	\forall\ a \in G\ \exists\ a^{-1} \in\ G s.t. a \cdot a^{-1} = i
\end{equation}

\subsection{Abelian Group}
A group is considered abelian if the operation is commutative

\begin{equation}
	\forall\ a,b \in G\ a \cdot b = b \cdot a
\end{equation}

\subsection{Theorems}

\subsection{Homomorphism}

Given two groups (G, $\ast$) and (H, $\cdot$) a group homomorphism is a function defined as h : G $\rightarrow$ H such that:
\begin{equation}
	\forall\ u,v \in G,\ h (u \ast v) = h (u) \cdot h(v)
\end{equation}
\subsection*{Kernel}
The kernel of a homomorphism is the set of elements in G that are mapped to the identity in H

\chapter{Ring Theory}
\section{Rings}

\subsection{Axioms}

The following ring axioms must be satisfied to qualify as a ring

\subsection*{Abelian group under addition}
\begin{equation}
	\forall\ a,b,c \in R\ (a + b) + c = a + (b + c)
\end{equation}
\begin{equation}
	a + b = b + a 
\end{equation}
\begin{equation}
	\exists\ 0\ \in R\ s.t.\ a + 0 = a
\end{equation}
\begin{equation}
	\exists -a\ \in R\ s.t.\ a + (-a) = 0
\end{equation}
\subsection*{Monoid under addition}
\begin{equation}
	\forall\ a,b,c \in R\ a \cdot (b \cdot c) = (a \cdot b) \cdot c
\end{equation}
\begin{equation}
	\exists\ 1 \in R\ s.t.\ a \cdot 1 = 1 \cdot a = a
\end{equation}
Existence of a unity element for multiplication is sometimes optional for definitions of a ring
\subsection*{Distributivity of multiplication with respect to addition}
\begin{equation}
	a \cdot (b + c) = a \cdot b + a \cdot c
\end{equation}
\begin{equation}
	(b + c) \cdot a = b \cdot a + c \cdot a 
\end{equation}
\subsection{Commutative Ring}
When the $\cdot$ operation is commutative then R is said to be a \textit{Commutative Ring} or \textit{Abelian Ring}
\subsection{Examples}

Infinite set of integers $\mathbb{Z}$

\subsection{Subring}

Let (R,+,x) be a ring and S a non-empty subset of R. If S still fulfills the ring axioms it is a sub-ring of R. R is also said to be a ring extension of S.

The trivial subring of R is S = \{0\}. If a subring is different to R it is called a proper sub-ring of R.

\subsection{Ideal}

Let R be a commutative ring, an Ideal is a non empty subset I.

( I , + ) is a sub-group of ( R , + )

I is stable if
\begin{equation}
	\forall\ i \in I, r \in R, i * r \in I
\end{equation}

I is a Proper Ideal if I $\neq$ R

\subsection*{Maximal Ideal}
For an ideal I of ring R
\begin{equation}
	s.t\ \nexists\ I' \subseteq I
\end{equation}

where I' is a distinct ideal of R

\subsection*{Principal Ideal}

\begin{equation}
	I = aR = \{ a * x\ |\ x \in R \}
\end{equation}



\subsection{Homomorphism}

https://www.youtube.com/watch?v=neQm8x0iJZk
\subsection{Quotient Ring}
Let I be a proper Ideal of R. The quotient group R/I is formed of elements of the cosets (r + I) of R with respect to I

\begin{equation}
	(r + I) \in I\ \{ r + i\ |\ i \in I \}
\end{equation}

Which can also form a quotient ring

\begin{equation}
	( r + I ) \oplus ( s + I ) = ( r + s + I)
\end{equation}
\begin{equation}
	( r + I ) \otimes ( s + I ) = ( r * s + I)
\end{equation}

(r $\in$ R, s $\in$ R) denoted by (R/I, $\oplus$, $\otimes$) or R/I 


\chapter{Field Theory}
\section{Fields}
A field is a set on which addition, subtraction, multiplication and division are defined and behave as the corresponding operations on rational and real numbers do

\subsection{Axioms}

The following field axioms must be satisfied to qualify as a field

\subsection*{Associativity of addition and multiplication}
\begin{equation}
	(a + b) + c = a + (b + c)\ \forall\ a,b,c \in F
\end{equation}
\begin{equation}
	(a \cdot b) \cdot c = a \cdot (b \cdot c)\ \forall\ a,b,c \in F
\end{equation}
\subsection*{Commutativity of addition and multiplication}
\begin{equation}
	a + b = b + a\ \forall\ a,b \in F 
\end{equation}
\begin{equation}
	a \cdot b = b \cdot a\ \forall\ a,b \in F 
\end{equation}
\subsection*{Additive and multiplicative identity}
\begin{equation}
	\exists\ c,d \in F,\ c \not\ \ d\ s.t.\ a + c = a\ ,\ b \cdot d = b
\end{equation}
\subsection*{Additive inverse}
\begin{equation}
	\exists\ a,a^{-1} \in F\ s.t.\ a + a^{-1} = 0
\end{equation}
\subsection*{Multiplicative inverse}
\begin{equation}
	\exists\ a,a^{-1} \in F\ s.t.\ a \cdot a^{-1} = 1
\end{equation}
\subsection*{Distributivity of multiplication over addition}
\begin{equation}
	\forall\ a,b,c \in F\ a \cdot (b + c) = a \cdot b + a \cdot c
\end{equation}

\end{document}